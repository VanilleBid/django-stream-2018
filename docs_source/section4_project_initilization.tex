\section{Initialization of the project}
    \emph{Note: change the current directory to the target parent project directory.}
        
    \subsection{Starting a new Django project}
        Create a new Django project by running the \texttt{django-admin} bootstrap, as follows: \\
        \code{django-admin startproject \emph{mysite}} \\
        
        This will create the following file structure:
        
        \begin{lstlisting}
|-- manage.py
|-- mysite
    |-- settings.py
    |-- urls.py
    |-- wsgi.py
    |-- __init__.py

1 directory
5 regular files
        \end{lstlisting}
        
        \begin{tabularx}{\linewidth}{l|X}
             \texttt{manage.py} &
                \href{https://docs.djangoproject.com/en/2.1/howto/custom-management-commands/}
                {This file allows you to run commands on the project, 
                like controlling the database or running the development HTTP server.} \\
             \texttt{mysite/settings.py} &
                \href{https://docs.djangoproject.com/en/2.1/topics/settings/}
                {The Django application settings file containing all your project configuration.} \\
             \texttt{mysite/urls.py} &
                \href{https://docs.djangoproject.com/en/2.1/topics/http/urls/}
                {A url mapping file to tell Django how to dispatch URLs. This file is also called the \emph{URLconf file}} \\
             \texttt{mysite/wsgi.py} &
                The project WSGI application file, used to serve the django application 
                with a production web server (nginx, Apache, etc.) \\
             \texttt{mysite/\textunderscore\textunderscore init\textunderscore\textunderscore.py} &
                The standard Python package file.
        \end{tabularx}
    
    \subsection{Optional: adding the dependencies to a file}
        \noindent Using any editor, create a \code{requirements.txt} file 
        containing the following content:

        \begin{lstlisting}[title=\texttt{requirements.txt}]
django~=2.1
%
        \end{lstlisting}
        
        This file allows you to directly install the requirements by running: \\
        \code{pip install -r requirements.txt}.
 