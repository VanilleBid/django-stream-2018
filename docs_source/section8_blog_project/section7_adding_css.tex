\subsection{Adding some CSS and extending the templates}
    Now, let's make our blog look better. For that, we are going to use the bootstrap framework.
    
    \subsection{Installing the bootstrap framework}
        We need to transform our \code{post\textunderscore list.html} to a valid 
        HTML template as done below. And then, we have to install Bootstrap 4
        in the \code{head} block of our HTML code.
        
        \begin{lstlisting}[
            language=html, 
            title=blog/templates/post\textunderscore list.html, escapechar=~]
<!doctype HTML>
<html>
  <head>
<link
rel="stylesheet"
href="//stackpath.bootstrapcdn.com/bootstrap/4.1.3/css/bootstrap.min.css"
/>
  </head>
  <body>
    <div class="container">
      
        <div>
          <h2>
            <a href="#">{{ post }}</a>
          </h2>
          <p class="text-muted">Date: {{ post.created_date }}</p>
          <p>{{ post.text | linebreaksbr }}</p>
        </div>
      
    </div>
  </body>
</html>
~
        \end{lstlisting}
    
    \subsection{Extending the templates}
        Later in our blog, we will have multiple 
        and different templates and pages: one to list the article, one to view a single article.
        One issue we will face with that is that we will have to repeat some base structure 
        code of our HTML template.
        
        The solution for that is extending a given base template and only adding whatever 
        new content we want.
        
        \subsubsection{Creating the base template}
            Let's create a \code{base.html} file that will contain the base HTML structure of~
            \code{post\textunderscore list.html}.
            
            \begin{lstlisting}[language=html, escapechar=~, title=blog/templates/base.html]
<!doctype HTML>
<html>
  <head>
<link
rel="stylesheet"
href="//stackpath.bootstrapcdn.com/bootstrap/4.1.3/css/bootstrap.min.css"
/>
  </head>
  <body>
    <div class="container">
      <div class="navbar">
        <span class="navbar-brand">
          <a href="">Blog Demo</a>
        </span>
      </div>
      
    </div>
  </body>
</html>
~
            \end{lstlisting}
            
            As you may note here, we removed the \texttt{for} loop that we had,
            so we removed everything that is used to show the post list. Instead,
            we replaced the content with \code{\{\% block content \%\}\{\% endblock \%\}}.
            
            The whole code of \code{base.html} will be extended from other templates
            (where \code{post\textunderscore list.html} is one of them). Then, we have
            the content block (\code{\{\% block content \%\}\{\% endblock \%\}})
            that will be overridden by the templates extending it to add their own content 
            to the base template.
            
        \subsubsection{Updating the post listing template}
            Now, we need to extend \code{base.html} and override the \code{content} block.
            
            \begin{lstlisting}[
                language=html, escapechar=~, 
                title=blog/templates/post\textunderscore list.html]


  
    <div>
      <h2>
        <a href="#">{{ post }}</a>
      </h2>
      <p class="text-muted">Date: {{ post.created_date }}</p>
      <p>{{ post.text | linebreaksbr }}</p>
    </div>
  

~
            \end{lstlisting}
