\subsection{The Forms}
    \label{sec:the-forms}
    Now that we have a page for viewing a single post, 
    and comments models and templates up and running,
    we can create a form allowing the user to add comments to our posts.
    
    \subsection{The ModelForm}
        In a new file named \code{forms.py}, we are going to extend a base
        form class that we will use as our form builder. This base form builder
        class is called \code{ModelForm} because it takes a \code{model}
        \emph{
            meta attribute\footnote{
                \raggedright
                To learn about meta classes, you can take a look at
                https://jfine-python-classes.readthedocs.io/en/latest/metaclass-attribute.html
                \begin{center}\qrcode[height=2.5cm]{
                    https://web.archive.org/web/20180327162929/
                    https://jfine-python-classes.readthedocs.io/en/latest/metaclass-attribute.html}
                \end{center}
            }}
        telling the form builder around what type of data model to build.
        
        \begin{lstlisting}[language=python, title=blog/forms.py]
from django import forms
from .models import Comment


class CommentForm(forms.ModelForm):
    class Meta:
        model = Comment
        fields = ('author_name', 'content')
%
        \end{lstlisting}
        
        \begin{description}
            \item[At line 7,] we are telling that we want to use the \code{Comment} 
                model for the form.
            \item[At line 8,] we then tell we only want to use and only handle the
                \code{author\textunderscore name} and \code{content} field of \code{Comment},
                thus, we exclude any field that we don't want the user to be able to manipulate or inject.
        \end{description}
        
        That's it.
        
    \subsubsection{Integrating the form to the views}
        We need to update the code base of the \code{post\textunderscore single} view,
        to integrate the \code{CommentForm} form. Here is how we are going to proceed:
        \begin{description}
            \item[First, we construct,] we need to initialize \code{CommentForm} with:
                \begin{itemize}
                    \item The HTTP POST data or nothing (\code{None}) is there is no data;
                    \item A target \code{Comment} instance where the form will commit the changes.
                \end{itemize}
            \item[Then, we check,] we check if the form was submitted and is valid, using the 
                \code{is\textunderscore valid} method.
                \begin{description}
                    \item[We commit,] and redirect the user if the form \textbf{is valid};
                    \item[We render the form,] if the form is \textbf{not valid}.
                \end{description}
        \end{description}
        
        \begin{lstlisting}[language=python, title=blog/views.py, numbers=none]
def post_single(request, post_id: int):
    try:
        # Prefetch the comments and retrieve the given post
        post = Post.objects.prefetch_related('comments').get(pk=post_id)
    except Post.DoesNotExist:
        raise Http404(f'Post #{post_id} not found.')

    # create a base empty comment to put user input into;
    # this base comment will have has parent post the current viewing post.
    comment_instance = Comment(post=post)
    
    # we create the form from the HTTP POST data or pass None if no POST data;
    # and we pass our empty comment instance to be populated.
    comment_form = CommentForm(request.POST or None, instance=comment_instance)

    # check if the form is valid
    if comment_form.is_valid():
        # commit the changes if the form is valid
        comment_form.save(True)
        
        # and redirect the user back to the single post view
        return redirect('post-single', post_id=post_id)

    # pass the form to render
    context = {'post': post, 'comment_form': comment_form}
    return render(request, 'blog/post_single.html', context)
%
        \end{lstlisting}
        
    \subsubsection{Updating the templates}
        Now, we just have to render the form as a HTML form in the single post template.
        For that, we just have to write \code{\{\{ form \}\}} in the template, 
        it will render its \code{<label>} and \code{<input>} automatically (inline form).
        Or call \code{form.as\textunderscore p} to render the form wrapped around \code{<p>} 
        tags (vertical form).
        
        \emph{
            Note: there is also \code{form.as\textunderscore table} to wrap around \code{<tr>} tags
            and \code{form.as\textunderscore ul} to wrap around \code{<li>} tags.}
            
        Here is the final code (to add after the \code{\%\{ endwith \%\}} instruction):
        \begin{lstlisting}[language=html, numbers=none, title=blog/templates/post\textunderscore single.html]
<form method="post">
  {{ comment_form.as_p  }}
  <button class="btn btn-primary" type="submit">Submit</button>
</form>
        \end{lstlisting}
        
        \emph{Note: the code doesn't work as is, see the next section.}
        \newpage
    
    \subsubsection{The CSRF protection}
        You may notice that if you run the above code as is, you run into this error:
        \begin{figure}[H]
            \centering
            \boxed{
                \includegraphics[width=\textwidth]{.assets/django-csrf-error.png}
            }
            \caption{CSRF verification failure error with \texttt{DEBUG=True}.}
        \end{figure}
        
        The reason behind this error, is that by default, Django is enabling the CSRF protection
        which is here to defend your website against the Cross-site Scripting (XSS) vulnerability.
        To be short: CSRF is an attack where a malicious entity manage to tricks
        a victim into performing actions on behalf of the attacker (e.g.: clicking on a link)\footnote{
            More information at: https://www.acunetix.com/websitesecurity/csrf-attacks/}.
        
        And one of the most popular protection against this type of attack, is to require the user to 
        give back a randomly generated token that was given to the browser when getting the form.
        
        And, in our above form, we did not give the user such CSRF token. To fix that, we need
        to call the tag \code{csrf\textunderscore token} in our \code{<form>}'s body. 
        This will generate a hidden \code{<input>} field containing a CSRF token, 
        in addition to a cookie.
        
        So, we have to edit our code to this:
        \begin{lstlisting}[
            language=html, escapechar=\~, numbers=none, 
            title=blog/templates/post\textunderscore single.html]
<form method="post">
   <!-- instruction to generate a CSRF token -->
  {{ comment_form.as_p  }}
  <button class="btn btn-primary" type="submit">Submit</button>
</form>
        \end{lstlisting}
