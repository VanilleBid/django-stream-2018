\subsection{The Admin Site}
    Now, we can use one of the features of the Django framework to quickly give us a way
    to manage our newly created model (retrieve, create, update, delete).
    
    For this, add the following content to the file \code{blog/admin.py}:
    
    \begin{lstlisting}[language=python]
from django.contrib import admin
from .models import Post

admin.site.register(Post)
%
    \end{lstlisting}
    
    This will register our \code{Post} model to the admin page. You can now open your
    web browser to \url{http://127.0.0.1:8000/admin/}, and you should see the following page:
    
    \begin{figure}[H]
        \centering
        \includegraphics[width=6cm]{.assets/django-admin-login-page.png}
        \caption{The django-admin login page.}
    \end{figure}
    
    \subsubsection{Creating the admin credentials}
        To login, we need to create a new admin user (a super-user). To do that, 
        we need to run the following command: \code{python manage.py createsuperuser}
        and fill everything. You will get something like this:
        
        \begin{lstlisting}[numbers=none]
Username (leave blank to use 'myusername'): admin
Email address: admin@example.com
Password:
Password (again):
The password is too similar to the email address.
This password is too short. It must contain at least 8 characters.
This password is too common.
Bypass password validation and create user anyway? [y/N]: y
Superuser created successfully.
        \end{lstlisting}
        
        Then we should be able to login if you return on the admin page. And then we should see this:
        
        \begin{figure}[H]
            \centering
            \includegraphics[width=12cm]{.assets/django-admin-index.png}
            \caption{Our django-admin index page, with the blog post model.}
        \end{figure}
        
        You can now play around with the posts, like adding a few posts, editing them, deleting them, etc.
