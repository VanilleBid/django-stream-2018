\subsection{The URL configuration}
    Now that we have our view, we can route it. 
    For that, create a URLConf file for our \code{blog} package (\code{blog/urls.py}), containing:
    
    \begin{lstlisting}[language=python, title=blog/urls.py]
from django.urls import path
from . import views

urlpatterns = [
    path('', views.post_list, name='post-list')
]
%
    \end{lstlisting}

    \begin{itemize}
        \item We create a list of URL patterns;
        \item We route as root point our \code{views.post\textunderscore list} view 
            and we tell Django that we want to internally call it \code{post-list}.
            The name will only be visible for the developer, and not for the user,
            it will allow us later to tell Django that we are speaking about that URL and not another one.
    \end{itemize}

    \subsection{Telling Django about our package URLs}
        What we have done is not enough, if you test, Django doesn't detect our new URL.
        And that is because we need to tell our site (or our core) to include the URLs of our \code{blog} package.
        
        For that, we edit \code{mysite/urls.py} to:
        
        \begin{lstlisting}[language=python, title=mysite/urls.py]
from django.contrib import admin
from django.urls import path, include

urlpatterns = [
    path('admin/', admin.site.urls),
    
    # include the URLs of the 'blog' package
    path('', include('blog.urls'))
]
%
        \end{lstlisting}

        If you look closely, you will notice a new instruction:
        \inlinepython{path('', include('blog.urls'))}. 
        This instructs Django to route all of our blog URLs to the root point
        (this will be \code{http://127.0.0.1:8000/}).
        
        That's as simple as that.
