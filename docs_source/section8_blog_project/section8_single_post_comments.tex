\subsection{Viewing comments over a single blog post}
    For the \emph{section~\ref{sec:the-forms}} demo example, 
    we would like to prepare a little comments section feature into our blog.
    We are gonna add a new \code{Comment} model and clicking on the posts title
    will now redirect the user to the single view by updating the template, URLs and views.

    \subsubsection{Adding the new models}
        Let's create a \code{Comment} model with a \texttt{name}, 
        a \texttt{content} and a \texttt{created\textunderscore date} field. 
        We will append it to \code{models.py}.
        
        \begin{lstlisting}[language=python, title=blog/models.py]
class Comment(models.Model):
    post = models.ForeignKey(
        to=Post, related_name='comments', null=False, blank=False,
        on_delete=models.CASCADE)
    author_name = models.CharField(max_length=40)
    content = models.CharField(max_length=500)

    class Meta:
        ordering = ('-pk',)
%
        \end{lstlisting}
        
        Little notes on how the \code{ForeignKey} field works:
        \begin{description}
            \item[on\textunderscore delete,] this is the behaviour to adopt when the referenced 
                object is deleted.
                \begin{itemize}
                    \item \code{models.CASCACE} means that when the referenced object is deleted, 
                        it must also delete the objects that have references to it.
                    \item \code{models.PROTECTED} it forbids the deletion of the referenced 
                        object until the references are not removed.
                    \item \code{models.SET\textunderscore NULL} means to set the referenced 
                        objects to \code{NULL}.
                \end{itemize}
        \end{description}
    
        Now, generate the new migrations file by running like before:\\
        \code{python manage.py makemigrations blog} which should output:
        
        \begin{lstlisting}[numbers=none]
Migrations for 'blog':
  blog\migrations\0002_auto_20181023_2205.py
    - Create model Comment
    - Add field comments to post
        \end{lstlisting}
        
        Then, apply the migrations by running\\
        \code{python manage.py migrate blog}. Which, should output:
        
        \begin{lstlisting}[numbers=none]
Operations to perform:
  Apply all migrations: blog
Running migrations:
  Applying blog.0002_auto_20181023_2205... OK
        \end{lstlisting}
    
    \subsubsection{Updating the views}
        First, we want to add a view that will allow us to view a single 
        post entry. For that, we will update the \code{views.py} file with this content:
        
        \begin{lstlisting}[language=python, title=blog/views.py]
from django.http import Http404
from django.shortcuts import render
from .models import Post


def post_list(request):
    context = {'posts': Post.objects.all()}
    return render(request, 'blog/post_list.html', context)


def post_single(request, post_id: int):
    try:
        # Prefetch the comments and retrieve the given post
        post = Post.objects.prefetch_related('comments').get(pk=post_id)
    except Post.DoesNotExist:
        raise Http404(f'Post #{post_id} not found.')

    return render(request, 'blog/post_single.html', {'post': post})
%
        \end{lstlisting}
        
        First of all (at line 1), we imported \code{Http404} from \code{django.http}
        to allow us to throw HTTP 404 errors on demand.
        
        Then, at line 11, we define a new view called \code{post\textunderscore single} 
        taking a \code{post\textunderscore id} parameter of the type \code{int},
        it will allow us to pass the \code{post\textunderscore id} parameter from the URL.
        
        After that we tell the query set of the Django's ORM that it will have to
        prefetch the \code{Post} comments. This will allow Django to know in advance 
        that we will need the comments and allow us to make the SQL request in one go,
        instead of doing one at the beginning (here) and again later in the templates 
        code when we will want the comments. So, for example, locally, 
        it saves 1ms for the response time.
        
        Then, we say that we want to retrieve the post having as primary key the given
        post identifier.
        
        But, if the post does not exist (e.g. the user wanted the post number 99),
        it will raise the \code{Post.DoesNotExist} exception that we handle to raise
        a HTTP 404 error instead.
        
        Then, at line 18, we render the single post template with the post.
        
    \subsubsection{Updating the URLs}
        Now, we update \code{urls.py} to route a single post to our newly created
        \code{post\textunderscore single} view.
        
        \begin{lstlisting}[language=python, title=blog/urls.py]
from django.urls import path
from . import views

urlpatterns = [
    path('', views.post_list, name='post-list'),
    path('post/<int:post_id>/', views.post_single, name='post-single')
]
%
        \end{lstlisting}
        
        You may notice at line 6, that you have \code{post/<int:post\textunderscore id>/}, 
        this tells Django that we are expecting an integer parameter in the URL
        and that we want it to be passed to the view as \code{post\textunderscore id}.
        
    \subsubsection{Updating the templates}
        Now we can route our posts to their single post view, to do that, 
        we update the link in \code{post\textunderscore list.html} to this:
        
        \begin{lstlisting}[language=html, numbers=none, escapechar=~]
<a href="">{{ post }}</a>
        \end{lstlisting}
        
        This will tell Django to generate a URL to a single post with the 
        identifier of the currently proceeded post.

        Then, we want to create the template \code{post\textunderscore single.html} 
        to show a single post and show the comments. We will put that:
        
        \begin{lstlisting}[language=html, escapechar=~, title=blog/templates/post\textunderscore single.html]



  <div class="card">
    <div class="card-body">
      <h5 class="card-title">{{ post.title }}</h5>
      <p class="card-text">{{ post.content | linebreaksbr }}</p>
    </div>
  </div>

  
    
      <div class="card mt-4" id="comments">
        <div class="card-body">
          <h5 class="card-title">Comments</h5>
          <ul class="list-group list-group-flush">
            
              <li class="list-group-item">
                {{ comment.name }} {{ comment.content }}
              </li>
            
          </ul>
        </div>
      </div>
    
  

~
        \end{lstlisting}
