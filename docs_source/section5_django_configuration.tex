\section{Configuration of the Django project}
    \subsection{The basics}
        There are many available settings, and infinite ways to configure a base Django project.
        Some of them are:
        
        \begin{tabularx}{\linewidth}{l|X}
             \code{DEBUG} &
                A boolean that turns on/off the debug mode.
                Never deploy a site into production with \code{DEBUG} turned on. \\
             \code{ALLOWED\textunderscore HOSTS} &
                A list of strings representing the host/domain names
                that this Django site can serve.\newline
                Example: \code{ALLOWED\textunderscore HOSTS = ['localhost', '127.0.0.1']} \\
             \code{LANGUAGE\textunderscore CODE} &
                A string representing the language code for this installation, default is
                \code{'en-us'} for U.S. English.
        \end{tabularx}
        
        You can find out more at this URL: \url{https://docs.djangoproject.com/en/2.1/ref/settings/}.
        
        {\centering 
        
            \qrcode{https://docs.djangoproject.com/en/2.1/ref/settings/}
            
        }
        
    \subsection{Setting up the database(s)}
        You can setup databases using the \code{DATABASES} settings, a dictionary containing
        the settings for all databases to be used with Django.
        It is a nested dictionary whose contents map a database alias
        to a dictionary containing the options for an individual database.
        
        You must configure a \code{default} database among any (optional) additional databases.
        You can easily configure Django to use SQLite, MySQL, Postgres, etc.
        
        More information at this URL: \url{https://docs.djangoproject.com/en/2.1/ref/settings/#std:setting-DATABASES}.
        
        ~
        
        {\centering
        
            \qrcode{https://docs.djangoproject.com/en/2.1/ref/settings/#std:setting-DATABASES}
        
        }
        
        ~
        
        Example SQLite configuration:
        \begin{lstlisting}[language=python]
DATABASES = {
    'default': {
        'ENGINE': 'django.db.backends.sqlite3',
        'NAME': 'mydatabase'
    }
}
        \end{lstlisting}
        
        Example Postgres configuration:
        \begin{lstlisting}[language=python]
DATABASES = {
    'default': {
        'ENGINE': 'django.db.backends.postgresql',
        'NAME': 'mydatabase',
        'USER': 'mydatabaseuser',
        'PASSWORD': 'mypassword',
        'HOST': '127.0.0.1',
        'PORT': '5432'
    }
}
        \end{lstlisting}
